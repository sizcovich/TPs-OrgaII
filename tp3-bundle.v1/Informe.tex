\documentclass[10pt, a4paper]{article}
\usepackage[paper=a4paper, left=1.5cm, right=1.5cm, bottom=1.5cm, top=3.5cm]{geometry}
\usepackage[T1]{fontenc}
\usepackage[spanish]{babel}
\usepackage[utf8]{inputenc}
\usepackage{indentfirst}
\usepackage{fancyhdr}
\usepackage{a4wide}
\usepackage[dvipsnames,usenames]{color}
\usepackage{float}
\usepackage{amsmath}
\usepackage{listings}
\usepackage{listingsutf8}
\usepackage{graphicx}
\usepackage{amsfonts}
\usepackage{verbatim}
\usepackage{latexsym}
\usepackage{lastpage}
\usepackage[colorlinks=true, linkcolor=blue]{hyperref}
\usepackage{calc}

\newcommand{\f}[1]{\text{#1}}
\newcommand{\real}{\mathbb{R}}
\newcommand{\nat}{\mathbb{N}}
\newcommand{\eme}{\mathcal{M}}
\newcommand{\emeh}{\widehat{\mathcal{M}}}
\newcommand{\ere}{\mathcal{R}}

\sloppy

\setlength{\voffset}{-0.5cm}
\setlength{\hoffset}{0.7cm}
\setlength{\headsep}{0pt}
\setlength{\headheight}{0pt}
\setlength{\oddsidemargin}{-0.7in}
\setlength{\marginparwidth}{-0.5cm}
\setlength{\textwidth}{18cm}
\setlength{\footskip}{2pt}
\setlength{\topmargin}{0in}
\setlength{\textheight}{25cm}
\setlength{\fboxrule}{3pt}

\begin{document}
\thispagestyle{empty}
\begin{center}

\Huge{ \bf{UNIVERSIDAD DE BUENOS AIRES}}
\\
\LARGE{\bf{Facultad de Ciencias Exactas y Naturales}}
\\
\textbf{Departamento de Computaci\'on}
\\
\textbf{Organizaci\'on del Computador}
\vspace{2.0\baselineskip}
\end{center}


\begin{figure}[h] %[h] Aqui [b] para button [t] para top
\begin{center}
\includegraphics[width=100pt]{./image.jpeg}
\end{center}
\end{figure}
\begin{center}
\vspace*{0.7cm}

\huge{\bf TRABAJO PR\'ACTICO N\'UMERO 3}\\
\huge{Nombre de Grupo: Napolitana con Jam\'on y Morrones}
\vspace*{8cm}

\end{center}

\huge{\textbf{Alumnos:}}\\
\\
\vspace*{0.3cm}
\Large{\textsl{Izcovich, Sabrina} $|$ sizcovich@gmail.com $|$ LU 550/11}\\
\vspace*{0.3cm}
\Large{\textsl{L\'opez Veluscek, Matías} \hspace{0.1cm}$|$ milopezv@gmail.com $|$ 926/10}\\
\vspace*{0.3cm}
\vspace{0.5cm}
 
\newpage
%Pagina de titulo e indice
\thispagestyle{empty}
%\large{
\tableofcontents
%}
\newpage
\section{Introducci\'on}


\section{Desarrollo}

\subsection{Ejercicio 1}
{\textbf{Pregunta 1:}} ¿Qué ocurre si se intenta escribir en la fila 26, columna 1 de la matriz de video, utilizando el segmento de la GDT que direcciona a la memoria de video? ¿Por qué?\newline
Depende de cómo fue definido el segmento. Si el tamaño del segmento es igual al tamaño de la pantalla en bytes, entonces va a ocurrir un Segmentation Fault. En el caso en el que el segmento no sea del mismo tamaño de la pantalla, el caracter no va a ser visible.\newline

{\textbf{Pregunta 2:}} ¿Qué ocurre si no se setean todos los registros de segmento al entrar en modo protegido? ¿Es necesario setearlos todos? ¿Por qué?\newline


\subsection{Ejercicio 2}
{\textbf{Pregunta 3:}} ¿Cómo se puede hacer para generar una excepción sin utilizar la instrucción int? Mencionar al menos 3 formas posibles.\newline

{\textbf{Pregunta 4:}} ¿Cuáles son los valores del stack cuando se genera una interrupción? ¿Qué significan? (Indicar para el caso de operar en nivel 3 y nivel 0).\newline


\subsection{Ejercicio 3}
{\textbf{Pregunta 5:}} ¿Puede el directorio de páginas estar en cualquier posición arbitraria de memoria?\newline

{\textbf{Pregunta 6:}} ¿Es posible acceder a una página de nivel de kernel desde usuario?\newline

{\textbf{Pregunta 7:}} ¿Se puede mapear una página física desde dos direcciones virtuales
distintas, de manera tal que una esté mapeada con nivel de usuario y la otra a nivel de kernel? De ser posible, ¿Qué problemas puede traer?\newline

\subsection{Ejercicio 4}
\subsection{Ejercicio 5}
\subsection{Ejercicio 6}
\subsection{Ejercicio 7}

\section{Conclusi\'on}


\end{document}
