\documentclass[10pt, a4paper]{article}
\usepackage[paper=a4paper, left=1.5cm, right=1.5cm, bottom=1.5cm, top=3.5cm]{geometry}
\usepackage[T1]{fontenc}
\usepackage[spanish]{babel}
\usepackage[utf8]{inputenc}
\usepackage{indentfirst}
\usepackage{fancyhdr}
\usepackage{a4wide}
\usepackage[dvipsnames,usenames]{color}
\usepackage{float}
\usepackage{amsmath}
\usepackage{listings}
\usepackage{listingsutf8}
\usepackage{graphicx}
\usepackage{amsfonts}
\usepackage{verbatim}
\usepackage{latexsym}
\usepackage{lastpage}
\usepackage[colorlinks=true, linkcolor=blue]{hyperref}
\usepackage{calc}

\newcommand{\f}[1]{\text{#1}}
\newcommand{\real}{\mathbb{R}}
\newcommand{\nat}{\mathbb{N}}
\newcommand{\eme}{\mathcal{M}}
\newcommand{\emeh}{\widehat{\mathcal{M}}}
\newcommand{\ere}{\mathcal{R}}

\sloppy

\setlength{\voffset}{-0.5cm}
\setlength{\hoffset}{0.7cm}
\setlength{\headsep}{0pt}
\setlength{\headheight}{0pt}
\setlength{\oddsidemargin}{-0.7in}
\setlength{\marginparwidth}{-0.5cm}
\setlength{\textwidth}{18cm}
\setlength{\footskip}{2pt}
\setlength{\topmargin}{0in}
\setlength{\textheight}{25cm}
\setlength{\fboxrule}{3pt}

\begin{document}
\thispagestyle{empty}
\begin{center}

\Huge{ \bf{UNIVERSIDAD DE BUENOS AIRES}}
\\
\LARGE{\bf{Facultad de Ciencias Exactas y Naturales}}
\\
\textbf{Departamento de Computaci\'on}
\\
\textbf{Organizaci\'on del Computador}
\vspace{2.0\baselineskip}
\end{center}


\begin{figure}[h] %[h] Aqui [b] para button [t] para top
\begin{center}
\includegraphics[width=100pt]{./image.jpeg}
\end{center}
\end{figure}
\begin{center}
\vspace*{0.7cm}

\huge{\bf TRABAJO PR\'ACTICO N\'UMERO 2}\\
\huge{Nombre del Grupo: Napolitana con Jam\'on y Morrones}
\vspace*{7.5cm}

\end{center}

\huge{\textbf{Alumnos:}}\\
\\
\vspace*{0.3cm}
\Large{\textsl{Izcovich, Sabrina} $|$ sizcovich@gmail.com $|$ LU 550/11}\\
\vspace*{0.3cm}
\Large{\textsl{L\'opez Veluscek, Matías} \hspace{0.1cm}$|$ milopezv@gmail.com $|$ 926/10}\\
\vspace*{0.3cm}
\Large{\textsl{Tito, Matías Gonzalo} \hspace{0.37cm}$|$ matias.tito@gmail.com $|$ 437/06}
\vspace*{0.3cm}
\vspace{0.6cm}
 
\newpage
%Pagina de titulo e indice
\thispagestyle{empty}
\large{
\tableofcontents
}
\newpage
\section{Introducci\'on}
El objetivo de este trabajo pr\'actico fue experimentar utilizando el modelo de programaci\'on SIMD. Para ello, fue requerido implementar seis filtros para procesamiento de im\'agenes (Recortar, Halftone, Umbralizar, Colorizar, Efecto Plasma y Rotar) tanto en $C$ como en $Assembler$.\newline
Por otro lado, debimos analizar la performance de un procesador al hacer uso de las operaciones SIMD. Para ello, realizamos comparaciones de velocidad entre los dos tipos de implementaciones realizados utilizando la herramienta Time Stamp Counter (TSC) del procesador.
\section{Desarrollo}
\section{Resultados}

\section{Conclusi\'on}
%\lstinputlisting[language=C++]{main.cpp}

%\centerline{\large{\textbf{metodos.h}}}
%\lstinputlisting[language=C++]{metodos.h}

%\centerline{\large{\textbf{formulas.cpp}}}
%\lstinputlisting[language=C++]{formulas.cpp}

%\centerline{\large{\textbf{formulas.h}}}
%\lstinputlisting[language=C++]{formulas.h}

%\centerline{\large{\textbf{metodos.cpp}}}
%\lstinputlisting[language=C++]{metodos.cpp}

\section{Referencias}

%%un libro se referencia asi AUTOR. Año. Título; subtítulo. Edición. Lugar de publicación, editorial. Páginas o
%%volumen. (Serie comercial)

\begin{itemize}
\item{Manual de Intel}

\end{itemize}

\end{document}
